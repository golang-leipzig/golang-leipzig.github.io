\documentclass{article}
\pagestyle{empty}
\usepackage[utf8]{inputenc}
\usepackage[T1]{fontenc}
\usepackage{typewriter}
\usepackage{verse}
\newcommand{\attrib}[1]{\nopagebreak{\raggedleft\footnotesize #1\par}}

\usepackage{hyperref}
\hypersetup{
    pdftitle={Three Haiku about the Go programming language},
    pdfsubject={Software-generated poetry, prepared for Leipzig Gophers \#38},
    pdfauthor={MiscModels},
    pdfkeywords={haiku, llm}
}

\begin{document}

The following pieces were generated in the afternoon of November 21, 2023 at
Leipzig University Library with a 2017 Kaby Lake CPU. The SHA1 of the models
used are fe938a131f40 (llama2), 4280f7257e73 (falcon) and 1629f2a8a495
(zephyr). A human reviewed about 1000 snippets from six different models and
choose the following three for purely subjective reasons. System prompt used:
\underline{Task is to write a poem.}
\underline{Do not emit introductory text like 'Sure' and other chat.}
\underline{Just write the poem and stop}, prompt was: \underline{write a haiku
    about the go programming language}.

\vspace{1.2cm}

\begin{verse}
    In your code \\
    A single line can make you think \\
    About all of the errors you might see. \\
\end{verse}
\attrib{falcon}

\begin{verse}
    Go, the language of grace \\
    Compact, efficient, and fast \\
    Code flows like water \\
\end{verse}
\attrib{llama2}

\begin{verse}
    compilation speed, \\
    golang's strength in production, \\
    developer's delight. \\
\end{verse}
\attrib{zephyr}

\end{document}
